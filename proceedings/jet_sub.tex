\documentclass[11pt]{cernrep}
\usepackage{graphicx,epsfig}
\bibliographystyle{lesHouches}

\usepackage{cite}
\usepackage{amsmath}
\usepackage[colorlinks=True,citecolor=blue]{hyperref}

\usepackage{subfigure}

\newcommand{\GeV}{\,\mathrm{GeV}}
\newcommand{\TeV}{\,\mathrm{TeV}}
\newcommand{\ie}{i.e.\ }
\newcommand{\eg}{e.g.\ }
\newcommand{\order}[1]{{\cal O}\left(#1\right)}
\newcommand{\avg}[1]{\left\langle\smash{#1}\right\rangle}
\newcommand{\as}{\alpha_s}
\newcommand{\ycut}{y_{\text{cut}}}
\newcommand{\zcut}{z_{\text{cut}}}
\newcommand{\fcut}{f_{\text{cut}}}
\newcommand{\ftrim}{f_{\text{trim}}}
\newcommand{\Rtrim}{R_{\text{trim}}}
\newcommand{\rtrim}{r_{\text{trim}}}
\newcommand{\zprune}{z_{\text{prune}}}
\newcommand{\Rprune}{R_{\text{prune}}}
\newcommand{\rprune}{r_{\text{prune}}}
\newcommand{\e}{\varepsilon}
\newcommand{\cf}{C_{F}}
\newcommand{\ca}{C_{A}}
\newcommand{\nf}{n_{F}}
\newcommand{\MSb}{\overline{\rm MS}}
\newcommand{\W}{{\rm W}}
\newcommand{\TiTj}{{\bf T}_i \cdot {\bf T}_j}
\newcommand{\ord}{\mathcal{O}}
\newcommand{\gstrong}{g_s}
\newcommand{\muNP}{\mu_\text{NP}}
\newcommand{\amax}{a_2}
\newcommand{\amin}{a_1}
\newcommand{\tlambda}{\tilde{\lambda}}



\renewcommand{\d}{\mathrm{d}}

\newcommand{\SD}{SoftDrop\xspace}
\newcommand{\ttt}[1]{{\small\texttt{#1}}}
\newcommand{\fastjet}{\texttt{FastJet}\xspace}
\newcommand{\fjcontrib}{\texttt{fjcontribtJet}\xspace}

\usepackage{color}
\definecolor{darkgreen}{rgb}{0,0.5,0}
\definecolor{darkblue}{rgb}{0,0,0.7}
\definecolor{darkred}{rgb}{0.5,0,0.0}

\newcommand{\sm}[1]{\textbf{\color{darkgreen}  [#1 -- sm]}}


\begin{document}

\section{Jet Studies: Four decades of gluons\protect\footnote{Section coordinators: S.~Marzani and B.~Nachman}$^{,}$~\protect\footnote{Contributing authors: S.~Amoroso, P.~Azzurri, H.~Brooks, S.~Forte, P.~Gras, Y.~Haddad, J.~Huston, A.~Larkoski, M.~Le Blanc, P.~Loch, K.~Long, E.~Metodiev, D.~Napoletano, S.~Prestel, P.~Richardson, F.~Ringer, J.~Roloff, D.~Soper, G.~Soyez, V.~Theeuwes}}

Studies related to gluon jets have played a key role in particle and nuclear physics since their discovery at PETRA exactly (to the day!) \textbf{four decades} prior to the 2019 Les Houches workshop.  This section investigates gluon fragmentation at the LHC, covering nearly \textbf{four decades} in energy scales.  Low energy scales involving gluon (sub)jets are studied from the point of view of hadronization and Monte Carlo tuning.  Higher-order effects in parton shower programs are investigated using deep learning.  Gluon jet rejection is considered in the context of vector boson fusion/scattering processes.  One of the main studies at this Les Houches was a study about the usefulness of a gluon jet differential cross section measurement in the context of parton distribution functions.  Gluon jet identification was also briefly discussed for searches at the highest energies accessible at the LHC.

\subsection{Introduction}
\label{sec:jets:intro}

Jets are collimated sprays of hadrons that emerge from high energy quarks and gluons and are an important asset or significant nuisance in a a majority of collider particle physics analyses.  Understanding jets and their internal structure (jet substructure~\cite{Abdesselam:2010pt,Altheimer:2012mn,Altheimer:2013yza,Adams:2015hiv,Asquith:2018igt,Larkoski:2017jix}) will directly or indirectly address a variety of fundamental questions in particle and nuclear physics.  One of the first studies related to jet substructure occurred nearly four decades ago, with the direct discovery of the gluon at PETRA~\cite{Brandelik:1979bd,Barber:1979yr,Berger:1979cj,Bartel:1979ut,Ellis:2014rma}.  It was of paramount importance at the time to study differences between jets initiated by quarks (quark jets) and jets initiated by gluons (gluon jets) in order to categorize the properties of the new boson.  This complex topic is still an active area of research in the present day and was the subject of the 2015 Les Houches report on jets~\cite{Badger:2016bpw,Gras:2017jty}.   The goal of this report is to study gluon jets at all relevant energies at the LHC, from non-perturbative scales all the way to the highest accessible energies.   Traversing nearly four decades in energy scales will reveal a plethora of interesting phenomena.  

At the lowest energies, jets are dominated by non-perturbative effects.  While there has been significant progress in understanding jet formation when fixed-order or resummed perturbation theory is accurate, there has been much less progress outside these regions of phase space.  While such contributions are often small for most observables, they are relevant for a precision program involving hadronic final states.  One example is the determination of the strong coupling constant, $\alpha_s$, from hadronic event shapes~\cite{Abbate:2010xh,Hoang:2015hka,TheALEPHCollaboration2004,DELPHICollaboration1997,Abdallah:2004xe,Biebel:1999zt,Abbiendi:2004qz,Buskulic:1992hq}.   After lattice determinations, the most precise extractions of $\alpha_s$ use thrust and the $C$-parameter from $e^+e^-$ data.  One of the biggest challenges of this extraction is that the non-perturbative corrections are nearly degenerate with changes to $\alpha_s$~\cite{Abbate:2010xh}.  The 2017 Les Houches report on jets studied the possibility of using jet substructure at the LHC to determine $\alpha_s$~\cite{Bendavid:2018nar}.  A key ingredient to this study is jet grooming, which is a set of tools to systematically remove soft and wide angle radiation within a jet.  Well-designed grooming algorithms allow for precise theory predictions of certain observables in part because non-perturbative effects are power suppressed.  While jet grooming may not be enough to eliminate the need to estimate non-perturbative effects, grooming may provide a unique opportunity to isolate these effects for further study.   While the perturbative regions of phase space have received significant attention from the community~\cite{Frye:2016aiz,Frye:2016okc,Marzani:2017mva,Marzani:2017kqd,Kang:2018vgn,Kang:2018jwa,Baron:2018nfz,Kardos:2018kth}, the non-perturbative regions have only recently been investigated~\cite{Hoang:2019ceu}.   One of the goals of this report is to explore the non-perturbative region of groomed jets using phenomenological tools for guidance. 

Both perturbative and non-perturbative regions of phase space at low energy can be important inputs to Parton Shower Monte Carlo (PSMC) parameter tuning.  In particular, there is a need for data enriched in gluon jets as many of the existing tunes are either based solely on or are anchored based on $e^+e^-$ data.  While those data are free from many nuisances like the underlying event, they are dominated by quark jets.  Various tuning campaigns at the LHC have found potential sources of tension between tunes that use jet substructure from the LHC and those that use jet and event shapes from LEP~\cite{ATL-PHYS-PUB-2014-021,Aad:2016oit}.  It is therefore critical to collect new measurements with unique and overlapping sensitivity to a variety of phase space regions.  The community repository for storing measurements is HepData~\cite{Buckley:2010jn,Maguire:2017ypu} and the standard for encoding an analysis for reinterpretation is Rivet~\cite{Buckley:2010ar}.  In the preparation of this report, new routines have been added to the existing databases and a list of jet substructure measurements from the LHC experiments has been tabulated.

While many aspects of PSMC programs are built on phenomenological models that must be tuned to data, there are also a variety of components that are based on fundamental aspects of the strong force and can be systematically improved.  Various MC programs such as \textsc{Dire}, \textsc{Vincia}, and \textsc{Deductor} include various subleading resummation, helicity, and color corrections.  In particular, the \textsc{Dire} program, which is a plugin to both \textsc{Pythia} or \textsc{Sherpa} now includes all of the next-to-leading order components of the QCD splitting functions including the triple-collinear and double-soft splittings.   The 2017 Les Houches report briefly reported an investigation of standard jet substructure observables (such as the two-prong tagger $N_2$~\cite{}) to the triple collinear splitting function~\cite{Bendavid:2018nar}.  A non-exhaustive list of such observables showed no sensitivity to this splitting function.  In order to know if any jet observable is sensitive to the extended physics modeling, deep neural network classifiers were constructed using the full observable jet phase space (kinematics and particle types).   This study confirmed that the triple-collinear splitting function is essentially non-observable, but the neural networks were able to significantly detect the double-soft splitting function.  Future work is required to construct simple observables that may become near-future measurements for probing this in data.

Jet classification techniques have been used for a variety of other tasks, including quark versus gluon (q/g) jet tagging.  In the context of Les Houches 2019, the focus of q/g tagging was on the isolation of vector boson fusion (VBF) and vector boson scattering (VBS) processes.  These processes are distinguished in part by two moderate $p_T$ forward quark jets.  In the context of Higgs production, quark/gluon tagging is useful both for separating the Higgs from other Standard Model backgrounds as well as separating different Higgs production modes.   The usefulness of q/g tagging to distinguish the gluon fusion (ggH) and VBF Higgs production modes was first investigated by CMS~\cite{}.  This report will show additional studies to understand the interplay between q/g tagging and other analysis selections such as requiring a large dijet invariant mass ($m_{jj}$).

While q/g tagging has traditionally been used to \textit{reject quarks}, there may also be a physics case for tagging selecting gluons.  One possibility in particular is the possibility of using gluon jets to constrain the gluon parton distribution function (PDF).  The gluon PDF has a large uncertainty at high $x$ ($m_{jj}\sim 1$ TeV) because the existing inclusive jet data are dominated by $qq$ initial states and $gg$ constraints from $t\bar{t}$ production become statistically limited.  What if one could directly measure the $gg$ reaction cross section?  The first step in answering this question is to establish a strong correlation between the initial and final state flavors.  This means that gluon tagging the final state will bias the initial state to be more gluonic.   The second step is to identify the tradeoff between tagging performance and uncertainties.  If the current PDF uncertainty can be made larger than the statistical and systematic uncertainties, new data would be useful in constraining the gluon PDF.  Lastly, it is important that the gluon tagging strategy is theoretically well-understood so that it can be simultaneously calculated with the $p_T$ spectrum as input to the PDF fits.  For this purpose, a new observable is considered - the \textit{Les Houches multiplicity} $n_\text{LH}$, first proposed in Ref.~\cite{Marzani:2019hun} as a variation on the iterative soft drop multiplicity~\cite{Frye:2017yrw}.

At the kinematic limit of LHC jets, gluon tagging may also have an important role for searches for new particles.  While not studied extensively at Les Houches, there was a general brainstorming session for gluon tagging applications and one promising example is the search for $X\rightarrow gg$.  At high $m_{jj}$, the SM background is dominated by valence quark scattering.  Furthermore, gluon tagging is more useful at high $p_T$ where counting observables like $n_{LH}$ have a better q/g tagging performance.  For these reasons, gluon tagging has an interesting potential to increase the sensitivity of the high mass dijet search.
%https://arxiv.org/pdf/1912.03511.pdf

This remainder of this chapter is organized from low to high energy3.  Section~\ref{sec:jets:np} begins with studies related to non-perturbative aspects of jets after grooming.  Then, Sec.~\ref{sec:jets:mc} investigates the potential for jet substructure observables for PSMC tuning.  Next, section~\ref{sec:jets:psmc} presents methods for probing higher-order effects in PSMCs.  A brief study of q/g tagging in the context of VBF/VBS is highlighted in Sec.~\ref{sec:jets:vbsbvf}.  At higher energies, the feasibility of Suppressing QUarks in the Region of RElatively Large-$x$ (\textsc{Squirrel}) is studied for the gluon PDF in Sec.~\ref{sec:jets:pdf}.  The kinematic limit is briefly described in Sec.~\ref{sec:jets:highest} and the chapter ends with conclusions and outlook in Sec.~\ref{sec:jets:conclusion}.

\subsection{Non-Perturbative effects at low jet mass}
\label{sec:jets:np}
(Ben, Simone, Jennifer, Eric, Vincent, Helen)

Soft drop/mMDT~\cite{Larkoski:2014wba,Dasgupta:2013ihk}, ATLAS~\cite{Aaboud:2017qwh,Aad:2019vyi} and CMS~\cite{Sirunyan:2018xdh}.  Analytics~\cite{Hoang:2019ceu}.  Resummation and fixed order~\cite{Frye:2016aiz,Frye:2016okc,Marzani:2017mva,Marzani:2017kqd,Kang:2018vgn,Kang:2018jwa,Baron:2018nfz,Kardos:2018kth}

\begin{figure}[h!]
\centering
\includegraphics[width=0.5\textwidth]{figs/Lowmassbump.pdf}
\caption{Figure adapted from Ref.~\cite{Frye:2016aiz}.  Replace me with PDF!}
\label{fig:jets:np:illustration}
\end{figure}

\subsection{Monte Carlo tuning with jet substructure observables}
\label{sec:jets:mc}
(Simone, Jennifer, Matt)

Rivet~\cite{Buckley:2010ar}, Professor~\cite{Buckley:2009bj}, HepData~\cite{Buckley:2010jn,Maguire:2017ypu}.  Table with routines (and new ones from this workshop).  See \href{https://twiki.cern.ch/twiki/bin/view/LHCPhysics/LHCJetSubstructureMeasurements}{this twiki}.

\subsubsection{Jet pull}
\label{sec:jets:pull}
(Helen, Vincent, Peter R)

Jet pull~\cite{Gallicchio:2010sw} and measurements~\cite{Aad:2015lxa,Aaboud:2018ibj}.

\subsection{Probing higher-order effects in PSMC}
\label{sec:jets:psmc}

\subsubsection{Triple Collinear Splitting Functions}
(Ben, Eric, Stefan Prestel)

\input{triplecollinearNN.tex}



Events are treated as sets of particles, with each particle $p_i$ specified by its momentum $\vec p_i^\mu$, mass, and particle-type.
%
The events are rotated to a consistent orientation by vertically aligning the second moment of the energy flow~\cite{Komiske:2019asc}.
%
This is accomplished by diagonalizing the spatial component of $\mathcal I^{\mu\nu} = \sum_{i=1}^M E_i v_i^\mu v_i^\nu$, where $v_i^\mu = p_i^\mu/E_i$ is the particle velocity.


As a machine learning architecture to process the entire events in their natural representation as sets of particles, we use Particle Flow Networks (PFNs)~\cite{Komiske:2018cqr} (see also \Ref{DBLP:conf/nips/ZaheerKRPSS17}).
%
Intuitively, PFNs learn a collection of additive observables which are processed by a fully-connected network.
%
A PFN acts on an event with $M$ particles $p_i$ as $\text{PFN}(\{p_i\}_{i=1}^M) = F\left(\sum_{i=1}^M \Phi(p_i)\right)$, where $F$ and $\Phi$ are parameterized by dense networks.
%
The network sizes of $F$ and $\Phi$ are identical to those in \Ref{Komiske:2018cqr}, with a latent space dimension of 256.
%
The train, validation, and test set sizes were 175k, 10k, and 15k, respectively.
%
The PFN classifiers were trained for 25 epochs with a batch size of 500.


\subsubsection{$g\to b \bar b$}
\label{sec:jets:gbb}
(Helen, Davide)

\subsection{q/g tagging in VBF and VBS}
\label{sec:jets:vbsbvf}
(Ben, Yachine, Kenneth, Paolo)



\subsection{\textsc{Squirrel} for the Gluon PDF}
\label{sec:jets:pdf}

\input{gluon_pdf.tex}

\subsection{The highest energy gluons at the LHC}
\label{sec:jets:highest}
(Ben)

\subsection{Conclusion and Outlook}
\label{sec:jets:conclusion}
(Simone)

\subsection*{Acknowledgments}

We thank the participants of Les Houches 2019 for a lively environment and useful discussions.
%%
BN is supported in part by the Office of High Energy Physics of the U.S. Department of Energy under Contract No. DE-AC02-05CH11231.
%
SM is also supported by the curiosity-driven grant "Using jets to challenge the Standard Model of particle physics" from Universit\`a di Genova.

\bibliography{lh2019}

\end{document}
